\chapter*{Введение}
Веб приложения сегодня являются одним из самых распространённых
способов создания сервисов, которые используются миллионами 
пользователей. Веб приложения легко обновлять, переносить, и 
распространять - они не требуют установки специальных средств 
на рабочей машине пользователей, кроме веб браузера.

В этой книге мы рассмотрим как создавать большие и структурированные
веб приложения используя Python и библиотеку Flask. Мы рассмотрим
такие вещи как безопасность веб приложений, базы данных, веб формы
и структурирование больших веб приложений.

Данная книга предназначена для начинающих веб программистов, а также
для всех тех, кто желает познакомиться с веб программированием. Мы предполагаем,
что читатель уже ознакомился с нашей книгой об основах Python и алгоритмах.
Данную книгу можно скачать бесплатно здесь~\cite{kuptsov:python}.

\chapter{Компоненты}

Современное веб приложение состоит из множества компонентов: системы управления
базами данных, или СУБД (как NoSQL, так и SQL), веб форм, контроллеров и моделей,
клиентской части приложения, написанной на JavaScript, подсистемы безопасности и 
многочисленных пользовательских библиотек. В данной главе мы рассмотрим основные 
компоненты, которые мы будем использовать при построение нашего приложения во 
второй части нашей книги.

\section{Среда окружения}

Часто на рабочей машине может быть запущено несколько проектов 
одновременно. Каждый проект должен иметь свои зависимости и установленные
библиотеки. Для того, чтобы не было путаницы в библиотеках
на рабочей машине, для разработки, устанавливают виртуальное окружение, а 
также нужные библиотеки.

На дистрибутиве Linux Ubuntu 20.04 (мы предполагаем наличие Python 3.8.5)
среда может быть установлена следующим образом.

Для начала необходимо установить пакет, который позволит работать с виртуальным
окружением:

\begin{python}
sudo apt-get install python3-venv
\end{python}

Далее создадим окружение:

\begin{python}
$ python3 -m venv book_ml
\end{python}

После нужно активировать виртуальную среду разработки:

\begin{python}
$ source book_ml/bin/activate
\end{python}

Далее можно устанавливать необходимые библиотеки и они не будут пересекаться
с другими проектами:

\begin{python}
$ pip3 install pycryptodome
\end{python}

\section{Безопасность}

Безопасность в веб приложениях является одним из основных
вопросов. Популярность веб приложений и их повсеместность 
накладывает определённые требования к безопасности. В данной
главе мы познакомим читателя с такими атаками как Cross
Site Request Forgery (CSRF), неаутентифицированный и неавторизованный
доступ к ресурсам, неверный ввод данных, а также рассмотрим методы защиты 
от таких атак. И конечно, мы затронем тему шифрования канала от пользователя 
до веб сервера с помощью Secure Socket Layer (SSL).

\subsection{CSRF}

Очень часто можно послать запрос на сайт замаскированный как
сторонний и произвести какую либо транзакцию так, чтобы пользователь
этого не заметил. Например, пусть пользователь авторизовался на вашем
сайте, получил куки и не вышел из системы. Позже злоумышленник может
прислать пользователю на почту картинку с изображением, например, кошечки
и попросил перейти по ссылке. Ссылка же на самом деле ведёт к вашему
сайту и автоматически пошлёт все куки файлы с запросом. 

Это опасно тем, что если система не защищена от CSRF атак, то 
запрос выполнится и атакующий сможет изменить содержимое базы 
данных. Для того чтобы этот тип атаки не смог быть реализован,
необходимо в форме HTML вставлять некий токен безопасности. Вместе
с тем, тот же токен необходимо хранить в зашифрованном виде в куке 
файле. Если на сервер придет куки файл, и токен в нем будет отличаться
от токена, полученного в форме, то запрос стоит отвергнуть, так как 
он небезопасный.

Приведём пример того, как можно использовать CSRF защиту в Flask 
приложении.

Для начала установим библиотеку для работы с формами и сам Flask:

\begin{python}
$ pip3 install flask
$ pip3 install flask_wtf
\end{python}

Далее нужно сконфигурировать наше Flask приложение для работы с CSRF
защитой. Для этого вставим следующие строки в наше приложение. Сначала
объявим форму:

\begin{python}
from flask_wtf import Form

class TestForm(Form):
	pass
\end{python}

А затем объявим контроллер, в котором данная форма будет отрисовываться:

\begin{python}
from flask import Flask, render_template, redirect, url_for, jsonify
from flask_wtf.csrf import CSRFProtect
from forms import TestForm

app = Flask()

csrf = CSRFProtect(app)

app.config["SECRET_KEY"] = "gliNuryoc6";

@app.route("/")
def index():
    form = TestForm()
    return render_template("index.html", form=form)

app.run(port=8080, debug=True);
\end{python}

В веб файле index.html прописываем скрытое поле, которое будет содержать наш CSRF токен:

\begin{python}
<form method="post">
    {{ form.csrf_token }}
</form>
\end{python}

Если вы используете jQuery AJAX запросы то нужно с запросом посылать и
токен:

\begin{python}
<script type="text/javascript">
    var csrf_token = "{{ csrf_token() }}";

    $.ajaxSetup({
        beforeSend: function(xhr, settings) {
            if (!/^(GET|HEAD|OPTIONS|TRACE)$/i.test(settings.type) && !this.crossDomain) {
                xhr.setRequestHeader("X-CSRFToken", csrf_token);
            }
        }
    });
</script>
\end{python}

Если проверка верности токена будет нейдачной, Flask выбросит ошибку CSRFError. 
По умолчанию, Flask вернет HTTP с кодом 400 и объяснением причины ошибки. Если
вы хотите отправить своё сообщение об ошибке то стоит зарегистрировать соответствующий
обработчик ошибки:

\begin{python}
from flask_wtf.csrf import CSRFError

@app.errorhandler(CSRFError)
def handle_csrf_error(e):
    return render_template('csrf_error.html', reason=e.description), 400
\end{python}

Стоит заметить CSRF защита требует секретный ключ для подписи токена. По умолчанию Flask 
будет использовать SECRET\_KEY переменную для этих целей. Если же вы захотите использовать
отдельный ключ для этих целей, то можно установить переменную WTF\_CSRF\_SECRET\_KEY.
Требуется использовать достаточно большой ключ для шифрования - ключ, который будет
содержать достаточно энтропии и его будет невозможно угадать. Мы рекомендуем использовать
ключ не менее $192$ бит. Для генерации такого ключа можно выполнить следующую команду в 
консоли Ubuntu:

\begin{python}[ht]
$ apg -x 32 -m 32 -a 1
\end{python}

Стоит заметить, что данная программа генерирует ключ, где каждый символ представляет собой
примерно $6$ бит энтропии. Поэтому $32$ символа - это примерно $192$ бита.

\subsection{Конфигурация безопасности}

Множество настроек существует для Flask приложения. Сюда входят как настройки общего характера,
так и настройки относящиеся к безопасности вашего веб приложения. В данной главе мы рассмотрим
настройки, относящиеся к безопасности.

https://flask.palletsprojects.com/en/2.0.x/config/

%https://towardsdatascience.com/how-to-secure-your-machine-learning-app-with-csrf-protection-506c3383f9e5

\subsection{Аутентификация и авторизация}

\subsection{Проверка форм регулярными выражениями}

\subsection{Шифрование данных с SSL}

%https://blog.miguelgrinberg.com/post/running-your-flask-application-over-https

Шифрование потока данных от пользователя до веб сервера является важным атрибутом
современного веб приложения: все данные, которые передаются от пользователя до сервера
будут зашифрованными (особенно это важно для таких данных, как пароль, данные о кредитной карте, куки файлы, 
пользовательская история, и множество других). Что бы достичь этого достаточно включить
шифрование в Flask приложении следующим образом:

\begin{python}
from flask import Flask
app = Flask(__name__)

@app.route("/")
def hello():
    return "Hello, World!"

if __name__ == "__main__":
    app.run(ssl_context='adhoc')
\end{python}

Данный способ будет генерировать самоподписанные сертификаты каждый раз, когда будет
стартовать наше веб приложение. Для того, чтобы данный способ работал необходимо установить
дополнительную зависимость:

\begin{python}
$ pip3 install pyopenssl
\end{python}

Проблема заключается в том, что при каждом старте приложения генерируется самоподписанный 
сертификат. Что не совсем удобно. Альтернативой является генерация своего собственного 
самоподписанного сертификата и использование его при любом старте приложения. Для того,
чтобы сгенерировать сертификат достаточно выполнить следующую команду в командной строке
Ubuntu:

\begin{python}
openssl req -x509 -newkey rsa:4096 -nodes -out cert.pem -keyout key.pem -days 365
\end{python}

После выполнения команды у вас появится два файла cert.pem и key.pem. cert.pem - это
самоподписанный сертификат, а key.pem - это секретный ключ. Срок действия сертификата - 
365 дней, а сложность модуля - $4096$ бит. Далее настроим наше веб приложение для работы 
с сертификатом:

\begin{python}
from flask import Flask
app = Flask(__name__)

@app.route("/")
def hello():
    return "Hello, World!"

if __name__ == "__main__":
    app.run(ssl_context=('cert.pem', 'key.pem'))
\end{python}

Но существует и другая проблема: Самоподписанные сертификаты не нравятся браузерам.
При первом открытии приложения, браузер выдаст ошибку, что сертификат не является
доверенным и его валидность будет необходимо подтвердить, приняв сообщение об угрозе.
Для того, чтобы ваши сертификаты были доверенными браузером можно купить подписанный 
сертификат у провайдера безопасности. Альтернативой являеться компания LetsEncrypt
~\cite{letsencrypt}, которая позволяет бесплатно генерировать подписанные и доверенные сертификаты
сроком на $90$ дней. Для этого достаточно установить библиотеку, настроить DNS
имя (у вас должно существовать зарегестрированное доменное имя) и сконфигурировать 
сервер Nginx на порту $80$. Начнем с установки библиотек и приложения certbot:

\begin{python}
$ sudo apt-get install software-properties-common
$ sudo add-apt-repository ppa:certbot/certbot
$ sudo apt-get update
$ sudo apt-get install certbot
\end{python}

Далее необходимо сконфигурировать Nginx сервер следующим образом:

\begin{python}
server {
    listen 80;
    server_name strangebit.io;
    location ~ /.well-known {
        root /vaw/www/certbot;
    }
    location / {
        return 301 https://$host$request_uri;
    }
}
\end{python}

И наконец мы можем запросить сертификат для нашего домена:

\begin{python}
$ sudo certbot certonly --webroot -w /vaw/www/certbot -d strangebit.io
\end{python}

Безусловно, домен $strangebit.io$ должен быть заменён вашим доменом. После 
успешного выполнения данной команды, ctrtibot создаст подписанный сертификат в 
$/etc/letsencrypt/live/strangebit.io/fullchain.pem$, а также секретный ключ
$/etc/letsencrypt/live/strangebit.io/privkey.pem$. Эти ключи уже можно будет 
использовать в вашем приложении. Но самое главное браузеры будут доверять 
таким сертификатам и атаки в виде Man-in-the-middle будут невозможными.

И на последок. Как создать $A+$ SSL сервер? После того, как были сгенерированы
сертификаты, остаются ещё пару тонкостей по настройке веб сервера: Первое, необходимо
создать параметры Diffie-Hellman, которые будут гарантировать необходимый уровень
безопасности, и второе, необходимо запретить небезопасные алгоритмы шифрования.

Для начала создадим параметры для Diffie-Hellamn. Это может занять долгое время, но
после выполнения команды у вас будет гаранития того, что обмен ключами будет 
криптостойким:

\begin{python}
openssl dhparam -out /etc/security/dhparam.pem 2048
\end{python}

И наконец зададим алгоритмы, которые будут доступны при создании SSL соединения:

\begin{python}
    ssl_ciphers 'ECDHE-RSA-AES256-GCM-SHA384:ECDHE-ECDSA-AES256-GCM-SHA384::ECDHE-RSA-AES256-SHA384:ECDHE-ECDSA-AES256-SHA384:ECDHE-RSA-AES256-SHA:ECDHE-ECDSA-AES256-SHA:DHE-RSA-AES256-SHA256:DHE-DSS-AES256-SHA:DHE-RSA-AES256-SHA:AES256-GCM-SHA384:AES128-SHA256:AES256-SHA256:AES256-SHA:AES:CAMELLIA:!DES-CBC3-SHA:!aNULL:!eNULL:!EXPORT:!DES:!RC4:!MD5:!PSK:!aECDH:!EDH-DSS-DES-CBC3-SHA:!EDH-RSA-DES-CBC3-SHA:!KRB5-DES-CBC3-SHA';
\end{python}

Тогда конфигурация Nginx будет выглядеть следующим образом:

\begin{python}
server {
    listen 443 ssl;
    server_name strangebit.io;
    ssl_certificate /etc/letsencrypt/live/strangebit.io/fullchain.pem;
    ssl_certificate_key /etc/letsencrypt/live/strangebit.io/privkey.pem;
    ssl_dhparam /etc/security/dhparam.pem;
    ssl_ciphers 'ECDHE-RSA-AES256-GCM-SHA384:ECDHE-ECDSA-AES256-GCM-SHA384::ECDHE-RSA-AES256-SHA384:ECDHE-ECDSA-AES256-SHA384:ECDHE-RSA-AES256-SHA:ECDHE-ECDSA-AES256-SHA:DHE-RSA-AES256-SHA256:DHE-DSS-AES256-SHA:DHE-RSA-AES256-SHA:AES256-GCM-SHA384:AES128-SHA256:AES256-SHA256:AES256-SHA:AES:CAMELLIA:!DES-CBC3-SHA:!aNULL:!eNULL:!EXPORT:!DES:!RC4:!MD5:!PSK:!aECDH:!EDH-DSS-DES-CBC3-SHA:!EDH-RSA-DES-CBC3-SHA:!KRB5-DES-CBC3-SHA';
    ssl_protocols TLSv1.3;
    ssl_session_timeout 1d;
    ssl_session_cache shared:SSL:50m;
    ssl_stapling on;
    ssl_stapling_verify on;
    add_header Strict-Transport-Security max-age=15768000;
}
\end{python}

\section{SQL: Модель данных}

\subsection{Объектно-ориентированный подход к запросам}

\subsection{Миграции}

\section{NoSQL}

\subsection{MongoDB}

\subsection{Cassandra}

\section{Представления}

\subsection{Jinja}

\subsection{Веб формы}

\subsection{Bootstrap}

\section{Контроллеры и бизнес логика}

\subsection{Blueprints: Структурируем большое приложение}

\section{Клиентский код: Vue.js}

\chapter{Разрабатываем приложение на примере машинного обучения}

