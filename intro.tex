\chapter*{Введение}
Веб приложения сегодня являются одним из самых распространённых
способов создания сервисов, которые используются миллионами 
пользователей. Веб приложения легко обновлять, переносить, и 
распространять - они не требуют установки специальных средств 
на рабочей машине пользователей, кроме веб браузера.

В этой книге мы рассмотрим как создавать большие и структурированные
веб приложения используя Python и библиотеку Flask. Мы расмотрим
такие вещи как безопасность веб приложений, базы данных, веб формы
и структурирование больших веб приложений.

Данная книга предназначена для начинающих веб программистов, а также
для всех тех, кто желает познакомиться с веб программированием.

\chapter{Компоненты}

\section{Среда окружения}

\section{Безопасность}

Безопасность в веб приложениях является одним из основных
вопросов. Популярность веб приложений и их повсеместность 
накладывает определённые требования к безопасности. В данной
главе мы познакомим читателя с такими атаками как Cross
Site Request Forgery (CSRF), неаутентифицированный и неавторизованный
доступ к ресурсам, а также рассмотрим методы защиты от таких атак. 
И конечно, мы затронем тему шифрования канала от пользователя до веб 
сервера с помощью Secure Socket Layer (SSL).

\subsection{CSRF}

\subsection{Аутентификация и авторизация}

\subsection{Шифрование данных с SSL}

\section{Базы данных}

\subsection{MySQL и SQLAlchemy}

\subsection{MongoDB и NoSQL}

\section{Веб формы}

\section{Jinja}

\section{Blueprints: Структурируем большое приложение}

\section{Bootstrap}

\section{Клиентский код: Vue.js}

\chapter{Разрабатываем блог}

