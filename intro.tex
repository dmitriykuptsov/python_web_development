\chapter*{Введение}
Веб приложения сегодня являются одним из самых распространённых
способов создания сервисов, которые используются миллионами 
пользователей. Веб приложения легко обновлять, переносить, и 
распространять - они не требуют установки специальных средств 
на рабочей машине пользователей, кроме веб браузера.

В этой книге мы рассмотрим как создавать большие и структурированные
веб приложения используя Python и библиотеку Flask. Мы расмотрим
такие вещи как безопасность веб приложений, базы данных, веб формы
и структурирование больших веб приложений.

Данная книга предназначена для начинающих веб программистов, а также
для всех тех, кто желает познакомиться с веб программированием.

\chapter{Компоненты}

\section{Среда окружения}

\section{Безопасность}

Безопасность в веб приложениях является одним из основных
вопросов. Популярность веб приложений и их повсеместность 
накладывает определённые требования к безопасности. В данной
главе мы познакомим читателя с такими атаками как Cross
Site Request Forgery (CSRF), неаутентифицированный и неавторизованный
доступ к ресурсам, а также рассмотрим методы защиты от таких атак. 
И конечно, мы затронем тему шифрования канала от пользователя до веб 
сервера с помощью Secure Socket Layer (SSL).

\subsection{CSRF}

Очень часто можно послать запрос на сайт замаскированный как
сторонний и произвести какую либо транзакцию так, чтобы пользователь
этого не заметил. Например, пусть пользователь авторизовался на вашем
сайте, получил куки и не вышел из системы. Позже злоумышленник может
прислать пользователю на почту картинку с изображением, например, кошечки
и попросил перейти по ссылке. Ссылка же на самом деле ведёт к вашему
сайту и автоматически пошлёт все куки файлы с запросом. 

Это опасно тем, что если система не защищена от CSRF атак, то 
запрос выполнится и атакующий сможет изменить содержимое базы 
данных. Для того чтобы этот тип атаки не смог быть реализован,
необходимо в форме HTML вставлять некий токен безопасности. Вместе
с тем, тот же токен необходимо хранить в зашифрованном виде в куке 
файле. Если на сервер придет куки файл, и токен в нем будет отличаться
от токена, полученного в форме, то запрос стоит отвергнуть, так как 
он не безопасный.

Приведём пример того, как можно использовать CSRF защиту в Flask 
приложении:

\begin{python}
from flask_wtf.csrf import CSRFProtect

csrf = CSRFProtect(app)
\end{python}

В веб форме прописываем скрытое поле, которое будет содержать наш CSRF токен:

\begin{python}
<form method="post">
    {{ form.csrf_token }}
</form>
\end{python}

Если вы используете jQuery AJAX запросы то можно с запросом посылать и
токен:

\begin{python}
<script type="text/javascript">
    var csrf_token = "{{ csrf_token() }}";

    $.ajaxSetup({
        beforeSend: function(xhr, settings) {
            if (!/^(GET|HEAD|OPTIONS|TRACE)$/i.test(settings.type) && !this.crossDomain) {
                xhr.setRequestHeader("X-CSRFToken", csrf_token);
            }
        }
    });
</script>
\end{python}

\subsection{Аутентификация и авторизация}

\subsection{Шифрование данных с SSL}

\section{Базы данных}

\subsection{MySQL и SQLAlchemy}

\subsection{MongoDB и NoSQL}

\section{Веб формы}

\section{Jinja}

\section{Blueprints: Структурируем большое приложение}

\section{Bootstrap}

\section{Клиентский код: Vue.js}

\chapter{Разрабатываем блог}

